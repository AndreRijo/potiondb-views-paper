\documentclass[sigconf, nonacm]{acmart}


\begin{document}

%\showthe\columnwidth
%\showthe\textwidth
%\showthe\linewidth

\title{Global Views on Partially Geo-Replicated Data}

%%
%% The "author" command and its associated commands are used to define the authors and their affiliations.
%
%\author{André Rijo, Carla Ferreira, Nuno Preguiça}
%\affiliation{
%\institution{NOVA LINCS, FCT, Universidade NOVA de Lisboa}
%\country{Portugal}
%}

%I used this example provided by VLDB:
%\author{Wang Xiu Ying}
%\author{Zhe Zuo}
%\affiliation{%
%	\institution{East China Normal University}
%	\city{Shanghai}
%	\country{China}
%}
%\email{firstname.lastname@ecnu.edu.cn}

%Maybe the mails should be used separately too as in their example?
%\author{J\"org von \"Arbach}
%\affiliation{%
%	\institution{University of T\"ubingen}
%	\city{T\"ubingen}
%	\country{Germany}
%}
%\email{jaerbach@uni-tuebingen.edu}
%\email{myprivate@email.com}
%\email{second@affiliation.mail}

%Note: from VLDB's guidelines: "list every author in its own author tag. The authors can, as shown in the template sample, share the same affiliations, but every author deserves his/her own author tag! There may not be any Additional Authors section."

%\email{a.rijo@campus.fct.unl.pt}
%\email{{nuno.preguica,carla.ferreira}@fct.unl.pt}
%\email{a.rijo@campus.fct.unl.pt, \{nuno.preguica,carla.ferreira\}@fct.unl.pt}

\begin{abstract}


Geo-replication is a key technique for providing low latency and high
availability in global services. As the number of data centers increase, so does the replication cost of full replication, which 
makes partial replication an attractive approach.

This paper presents PotionDB, a novel geo-distributed, partially replicated key-value store.  
PotionDB transaction management and replication algorithms provide transactional causal consistency 
for transaction that access local and remote objects. 
For supporting recurrent queries over geo-partitioned data, PotionDB provides
efficient materialized views, that allow these queries to be replied locally at all replicas.
Our evaluation shows that queries concerning global data can be completed in less than one millisecond 
while maintaining high throughput, despite the extra operations required to keep views up-to-date. 


%Many existing web services at a global scale have strict latency, availability, and fault tolerance requirements.
%To address these requirements,  services are often deployed in data centers distributed 
%across the globe, with users accessing the closest data center.
%As the number of servers increases, so does the replication cost, which makes partial replication an attractive approach.
%However, some queries may require data not replicated in every server (e.g. in an e-commerce system, the best-selling products across the globe).
%
%In this paper, we present PotionDB, a novel geo-distributed key-value store that provides partial replication.
%We present a replication scheme that allows data to be partially replicated, yet still, answer queries relative to global data efficiently.
%We leverage existing partial and non-uniform replication algorithms to provide materialized views of data that may be split across multiple servers.
%With these views, a client can obtain global information by querying a single server.
%Our evaluation shows that queries concerning global data can be completed in less than one millisecond while maintaining high throughput, despite the extra operations required to keep views up-to-date. 


\end{abstract}

\maketitle

\section{Introduction}
\label{sec:introduction}

%	\item Num. DC a aumentar

The increasing reliance on web services in many domains of activity leads to stringent requirements regarding latency, availability,
and fault tolerance~\cite{Schurman2009latency,gomez}.
To address these requirements, cloud platforms have been adding new data centers at different geographic 
locations. By allowing users to contact the closest data center, a global service can
provide low latency to users spread across the globe. 
The increasing number of data centers also contributes
for providing high availability,  as a user can access any
available data center.


\balance

\end{document}
%\bibliographystyle{abbrv}
%\bibliographystyle{ACM-Reference-Format}
%\bibliography{bib}

