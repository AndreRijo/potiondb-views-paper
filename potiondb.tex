\documentclass{vldb}

\usepackage[utf8]{inputenc}
\usepackage{times}
\usepackage{graphicx}
\usepackage{xcolor}
\usepackage{hyperref}
\usepackage{paralist}
\usepackage[inline]{enumitem}



\usepackage{balance}  % for  \balance command ON LAST PAGE  (only there!)


\newcommand{\grumbler}[2]{{\color{red}{\bf #1:} #2}}
\newcommand{\andre}[1]{\grumbler{andre}{#1}}
\newcommand{\nuno}[1]{\grumbler{nuno}{#1}}
\newcommand{\carla}[1]{\grumbler{carla}{#1}}

\newcommand{\outline}[1]{}
%\newcommand{\outline}[1]{\grumbler{outline}{#1}}

\vldbTitle{}
\vldbAuthors{}
\vldbVolume{12}
\vldbNumber{xxx}
\vldbYear{2020}
\vldbDOI{https://doi.org/TBD}


\begin{document}

\title{Global Views on Partially Geo-Replicated Data}
\numberofauthors{3} %  in this sample file, there are a *total*
% of EIGHT authors. SIX appear on the 'first-page' (for formatting
% reasons) and the remaining two appear in the \additionalauthors section.

\author{\alignauthor André Rijo\\
       \affaddr{NOVA LINCS, FCT, Universidade NOVA de Lisboa}\\
%       \email{v.sousa@campus.fct.unl.pt}
\alignauthor Carla Ferreira\\
       \affaddr{NOVA LINCS, FCT, Universidade NOVA de Lisboa}\\
%       \email{carla.ferreira@fct.unl.pt}
\alignauthor
Nuno Preguiça\\
       \affaddr{NOVA LINCS, FCT, Universidade NOVA de Lisboa}\\
%       \email{nuno.preguica@fct.unl.pt}
}


\maketitle

\begin{abstract}
bla

bla

bla

bla

bla

bla

bla

bla

\end{abstract}

\section{Introduction}

%	\item Num. DC a aumentar

The increasing reliance on web service in many domains of activity, from e-commerce to business applications
and entertainment, leads to stringent requirements regarding latency, availability and fault tolerance \cite{Schurman2009latency,gomez}.
To address these requirements, cloud platforms have been adding new data centers at different geographic 
locations. By allowing users to access a service by contacting the closest data center, a global service can
provide low latency to users spread across the globe. The increasing number of data centers also contributes
for proving high availability and fault tolerance, by allowing a user to access the service by accessing any
available data center.

% serviços necessitam de data
% Replicar totalmente tem problemas

The database is a key component of any web service, storing the service's data. For supporting global
services running at multiple geographic locations, it is necessary to rely on a geo-replicated database \cite{dynamo},
which maintains replicas of the data at the data centers where the service is running.
A number of geo-replicated database have been proposed, providing different consistency semantics.
Databases that provide strong consistency \cite{spanner,cockroachdb,mdcc} intend to give  the illusion that 
a single replica exists, requiring coordination among
multiple replicas for executing (update) operations. This leads to high latency and may compromise 
availability int he presence of network partitions.
Database that provide weak consistency \cite{eventual,dynamo,cops} allow any replica to process a
client request, leading to lower latency and high availability. As a consequence, these databases expose
temporary state divergence to clients, making it more difficult to program a system. 

In either case, geo-replicated databases typically rely on a full replication model, where each data 
center replicates the full database, with data being sharded across multiple partitions in each data 
center. 
As both the data managed by these systems increases in size and the number of data centers increases,
this approach leads to a number of problems.
First, storing all data in all data centers imposes a large overhead in terms of storage. 
Furthermore, storing some data in all data centers may be unnecessary, as data is only needed at some
geographic locations.
Second, increasing the number of data centers makes the replications process more complex and costly, 
as each update needs to be propagated to all other data centers.

For addressing these problems, partial replication is an attractive approach, with each data center
replicating only a subset of the data. A number of works have been addressing the challenges of 
partial replication, for example by proposing algorithms to manage partially replicated data \cite{more,saturn,c3}
and to decide which data is replicated in which replica \cite{}.

In this paper we address the problem of querying data in a weakly consistent partially geo-replicated database, 
focusing on recurrent queries for which a programmer would want to generate a (materialized) view.
For example, consider an e-commerce system with users from multiple geographic locations.
In this case, the data pertaining users of a given location do not need to be replicated in all data centers
(but only in a few for fault tolerance). The same applies to other information, such as data on orders and 
warehouses.
Other data, such as information on products would be replicated in the regions where the product
is available.  
Under this data placement, obtaining the list of best seller products is challenging, as it requires
accessing data that is located at multiple data centers.

Several possible solutions exist for this problem. 
First, it is possible to have a data center that replicates all data, and forward these queries to such data center.
Doing this imposes a latency penalty and requires a data center to host all data and execute all queries of this type. 
Second, it is possible to execute the query by accessing multiple locations, by using, for example, 
a distributed processing system with support for geo-partitioned data \cite{Kloudas:2015:POD:2850578.2850582,more}.
This approach requires running an additional external service and poses challenges for the consistency of the results
returned and the data observed by users.
%, mostly when adopting weak consistency models (a local update that should
%be reflected in the result of the query may not be returned, as the update might have not been read by the 
%external service that accessed a different replica).

We propose a different approach: to maintain materialized views, as commonly available in relational databases.
Implementing such feature efficiently in a partially geo-replicated database requires 
addressing two main challenge challenges. 
First, it is necessary to guarantee consistency between the base data available in a replica and the 
relevant materialized views. To achieve this, we designed a replication mechanism where updates 
to the base data and views are made visible atomically in each replica.

Second, it is necessary to efficiently support views with limits, used for example to support \emph{top-k} 
queries. To achieve this, we build on the concept of non-uniform replication \cite{Cabrita17Nonuniform}, in which the state
of different replicas may be different, given that the observable state is (eventually) the same.
This allows each replica to propagate only the updates that might be relevant to the observable 
state. Providing support for views required us to extend non-uniform replication from simple data 
types to more complex structures that could support a view with multiple columns.
\nuno{can we support updates to the views? why not?}

We present the design and implementation of PotionDB, a geo-replicated key-value store with support  
for partial replication and materialized views. 
PotionDB provides weak consistency, for improved latency and availability, and support for highly
available transactions \cite{hat}.  
To our knowledge, our work is the first to address the problem of maintaining materialized views
in such setting.  

We have evaluated our system using micro-benchmarks and TPC-H queries \cite{} \nuno{TPC-H, certo?}.
The results show that our algorithms for maintaining materialized views impose 
low overhead when executing and asynchronously replicating transactions, particularly
for views with limits.
\nuno{deviamos ter uns micro-benchmarks que comparassem o overhead com limites e sem limites}
Additionally, the results show that executing queries by relying on the materialized views is much more 
efficient than using alternative mechanisms.  
Furthermore, our algorithms for maintaining materialized views in a decentralized way perform better 
that alternative approaches where the view is computed in a single data center, while being
able to  keep consistency between the base and view data in every replica.

In this paper we make the following contributions:
%\begin{enumerate*}[label=(\roman*)]
\begin{itemize}
	\item the design of a geo-replicated key-value store  with support for partial replication
	and views over partially replicated data; 
	\item replication algorithms for efficiently maintaining consistent materialized views over 
	partially replicated data;
	 \item an implementation and evaluation of the proposed approach with micro-benchmarks
	 and TPC-H.
\end{itemize}

The remainder of the paper is organized as follows. Section ...

\outline{topicos

\begin{itemize}
	\item Num. DC a aumentar
	\item Replicar totalmente tem problemas
	\item Replicação parcial
	\item Queries sobre dados replicados parcialmente
	\begin{itemize}
		\item Standard solution?
	\end{itemize}
	\item Views materializadas replicadas totalmente
	\item Contribuições
\end{itemize}
}

\section{System overview}

\begin{itemize}
	\item System model
	\begin{itemize}
		\item Replicação parcial
	\end{itemize}
	\item System API
	\begin{itemize}
		\item "Create table"
		\item "Create view"
		\begin{itemize}
			\item CRDT não uniforme
			\item put numa table $\implies$ puts nas várias views
			\begin{itemize}
				\item consistência das views face aos dados - in sync
			\end{itemize}
		\end{itemize}
	\end{itemize}
	\item System description
	\begin{itemize}
		\item CRDT não uniforme
		\item Implementação de queries?
	\end{itemize}

\end{itemize}

\section{Implementation}

\section{Evaluation}

\section{Related Work}

\section{Conclusions}


\bibliographystyle{abbrv}
\bibliography{bib}

\section{System overview}

\vspace{8cm}
Possiveis pontos mais detalhados?

\subsection{System model}

\begin{itemize}
	\item Network assumptions
	\item Client-server interaction (refer key-value store interface? Maybe refer this instead in System API?)
	\item Server-server interaction? (is it needed? We'll already touch this in Replication.)
	\item System guarantees
	\begin{itemize}
		\item CRDTs
		\item Consistency level	
	\end{itemize}
	\item Replication
		\item Async
		\item Op-based
		\item Maintains consistency, i.e., transaction level based.
		\item Partial (system admin defined, each server only has a subset of the data based on topics. Potencially some data can be replicated everywhere)
\end{itemize}

\subsection{System API}

\begin{itemize}
	\item Basically how can we translate a problem to sql-like operations
	\item Create table
	\item Create view
	\item Updates (incluir problema de consistência de views/dados)
	\item Queries (incluir aqui problema de os CRDTs não uniformes precisarem de mais dados? Ou na zona da view?)
\end{itemize}

\subsection{System description}

\begin{itemize}
\item Structure? Maybe that's for implementation? How much detail?
\begin{itemize}
	\item Internal partitioning vs external partitioning? Capaz de não ser boa ideia...	
\end{itemize}
\item CRDTs and non-uniform CRDTs?
\end{itemize}

\section{Implementation}

\begin{itemize}
	\item Go
	\item Transactions (TM/Mat?)
	\item Replication (RabbitMQ and other stuff?)
	\item Communication (protobufs. Also worth noticing the compability with existing AntidoteDB clients)
	\item CRDTs (version management at least)
\end{itemize}

\end{document}
